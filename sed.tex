\chapter{SED 行编辑器}
\section{sed的由来}
根据维基百科资料,Ken Thompson 是 UNIX 操作系统的共同创始人之一,他在1969年于贝尔实验室开发了最初的 UNIX
系统,并创造了早期的行编辑器 \shellinline{ed}。\shellinline{ed}
虽功能强大,但交互性差,不适合自动化处理。为弥补这一不足,Lee E. McMahon 于1973年在 UNIX 第四版中开发了 \shellinline{sed},其核心思想正是延续 Thompson 所倡导的“简洁、专注、可组合”原则。

\shellinline{sed} 的设计直接受到 \shellinline{ed} 命令语法的影响,但它专为非交互式、逐行处理文本流而优化,天然契合 UNIX
管道(pipe)机制。用户可将 \shellinline{sed} 与其他命令(如 \shellinline{grep}、\shellinline{awk}、\shellinline{cat})串联,实现高效的数据处理流水线——这正是 Thompson 与 Dennis Ritchie 所构建的 UNIX 工具链理念的体现:每个程序只做一件事,并做到极致。

值得一提的是,Thompson 不仅推动了操作系统架构的革新,还通过正则表达式等抽象机制为文本处理奠定了基础。\shellinline{sed}
正是这一思想的直接产物。因此,在学习 \shellinline{sed} 语法(如替换 \shellinline{s/pattern/repl/}、地址匹配等)的同时,了解其源自贝尔实验室、承袭自 Thompson 的工程美学,有助于我们不仅“会用”命令,更能理解 UNIX 为何能成为现代计算的基石。这种历史视角,让技术学习更具深度与人文温度。

