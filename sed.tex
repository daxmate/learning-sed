\chapter{SED 编辑器}
\section{肯·汤普森与 ed 编辑器}


肯·汤普森(Kenneth Lane Thompson),1943 年 2 月 4 日出生于美国路易斯安那州新奥尔良,是计算机科学领域的先驱人物。他于 1965 年和 1966 年在加州大学伯克利分校分别获得了电气工程与计算机科学的学士和硕士学位。1966 年,汤普森加入贝尔实验室,开启了他的传奇职业生涯。

\begin{wrapfigure}{l}{0.45\textwidth}
	\centering
	\includegraphics[width=0.28\textwidth]{images/thompson}
	\caption{Thompson 2019}
\end{wrapfigure}

在贝尔实验室,汤普森与丹尼斯·里奇(Dennis Ritchie)等人合作,参与了 Multics 操作系统的开发。在这一过程中,他创造了 Bon 编程语言,并开发了一款名为 Space Travel 的视频游戏。后来,贝尔实验室退出了 Multics 项目,但汤普森对编程的热情并未减退。他利用一台旧的 PDP-7 机器,将 Space Travel 重新编写,而在这个过程中,他和同事们开发的一系列工具逐渐演变成了 Unix 操作系统。

在 Unix 的开发过程中,汤普森意识到操作系统需要一种系统编程语言,于是他创造了 B 语言,这是里奇后来开发 C 语言的直接前身。而除了在编程语言和操作系统方面的贡献,汤普森在文本编辑器领域也有着重要的影响,他发明了 \shellinline{ed} 编辑器。

\shellinline{ed} 是 Unix 系统上的标准文本编辑器,它的出现极大地推动了文本处理的发展。\shellinline{ed} 编辑器以其简洁和高效而著称,它允许用户通过简单的命令来对文本文件进行编辑操作,如插入、删除、替换等。\shellinline{ed} 编辑器的命令语法简洁明了,虽然它没有现代文本编辑器那样丰富的图形用户界面,但在当时,它为用户提供了强大的文本处理能力,成为 Unix 系统中不可或缺的工具之一。

\shellinline{ed} 编辑器的出现,为后续文本编辑器的发展奠定了基础,其中就包括 \shellinline{sed} 编辑器。\shellinline{sed}(stream editor)是基于 \shellinline{ed} 的思想和语法发展而来的流编辑器,它在 \shellinline{ed} 的基础上进行了扩展和优化,能够对文本流进行更灵活、更高效的处理。\shellinline{sed} 编辑器的出现,进一步丰富了 Unix 系统中的文本处理工具,也为后来的文本处理工具的发展提供了重要的借鉴。

肯·汤普森的贡献不仅局限于 \shellinline{ed} 编辑器,他在计算机科学领域的诸多成就都对后世产生了深远的影响。他与里奇共同获得了 1983 年的图灵奖,以表彰他们在 Unix 操作系统开发方面的杰出贡献。此外,他还获得了众多其他荣誉和奖项,包括 IEEE Richard W. Hamming 奖章、美国国家技术奖章等,这些都充分证明了他在计算机科学领域的卓越地位。

汤普森的工作和创新精神激励着一代又一代的程序员和计算机科学家。他的贡献不仅推动了计算机技术的发展,也改变了人们使用计算机的方式。而 \shellinline{ed} 编辑器作为他众多成就中的一部分,虽然在现代文本编辑器的浪潮中逐渐被边缘化,但它在计算机文本处理历史上的重要地位不可磨灭,它为后续编辑器的发展提供了宝贵的思路经验,\shellinline{sed} 编辑器就是其中的杰出代表。
\section{李·爱德华·麦克马洪和sed}
\begin{wrapfigure}{r}{0.5\textwidth}
	\centering
	\includegraphics[width=0.28\textwidth]{images/mcmahon}
	\caption{McMahon}
\end{wrapfigure}
李·爱德华·麦克马洪(Lee Edward
McMahon)是贝尔实验室的计算机科学家。他在1973年至1974年期间开发了\shellinline{sed}编辑器,旨在创建一个通用的、基于行的流编辑器,以弥补\shellinline{ed}编辑器在处理大型文件或批量文本操作时的不足。在开发\shellinline{sed}之前,麦克马洪已经参与了多个文本编辑工具的开发工作,包括\shellinline{ed}和更早的\shellinline{qed},这些经验为他设计\shellinline{sed}提供了重要基础。

\shellinline{sed}(stream editor)是一个基于文本的流编辑器,主要用于解析和转换文本。它使用一种简单而紧凑的编程语言,能够对文本文件执行替换、删除、插入等操作。作为Unix系统中的标准工具,\shellinline{sed}以其简洁高效的语法而著称,支持正则表达式,能够灵活匹配文本模式。其流式处理方式逐行读取输入文本,应用命令后输出结果,这使得它能够高效处理大型文件。此外,\shellinline{sed}具有良好的可扩展性,可以与其他工具如AWK和Perl结合使用,以应对更复杂的文本处理任务。

\shellinline{sed}的开发源于对\shellinline{ed}编辑器局限性的改进。\shellinline{ed}作为交互式工具,在批量处理时操作繁琐,而\shellinline{sed}通过流式处理和命令批量化,显著提升了文本处理效率。\shellinline{sed}广泛应用于文本替换、过滤、格式化及自动化等场景,例如替换特定字符串、删除符合模式的行或调整文本格式。尽管\shellinline{sed}功能强大,但对于高度复杂的任务,它可能显得不够灵活,此时可以使用AWK或Perl等工具,这些工具提供了更丰富的编程结构和数据处理能力。

麦克马洪的创新为文本处理工具的发展奠定了坚实基础,\shellinline{sed}不仅成为Unix系统中不可或缺的工具,还影响了后续工具如AWK和Perl的设计。他的工作延续了贝尔实验室在计算机科学领域的传统,为高效文本处理提供了持久价值。

\section[ed的基本用法]{\shellinline{ed}的基本用法}
下面我们来演示一下\shellinline{ed}的基本用法,\shellinline{ed}由于设计的比较早,随着计算机性能的提升早已淡出实际使用的场景。这里的演示不是为了在实际的场景中使用它,而是了解它的设计思路以及其产生的背景。
